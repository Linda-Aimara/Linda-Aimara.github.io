% Options for packages loaded elsewhere
\PassOptionsToPackage{unicode}{hyperref}
\PassOptionsToPackage{hyphens}{url}
%
\documentclass[
]{article}
\usepackage{amsmath,amssymb}
\usepackage{iftex}
\ifPDFTeX
  \usepackage[T1]{fontenc}
  \usepackage[utf8]{inputenc}
  \usepackage{textcomp} % provide euro and other symbols
\else % if luatex or xetex
  \usepackage{unicode-math} % this also loads fontspec
  \defaultfontfeatures{Scale=MatchLowercase}
  \defaultfontfeatures[\rmfamily]{Ligatures=TeX,Scale=1}
\fi
\usepackage{lmodern}
\ifPDFTeX\else
  % xetex/luatex font selection
\fi
% Use upquote if available, for straight quotes in verbatim environments
\IfFileExists{upquote.sty}{\usepackage{upquote}}{}
\IfFileExists{microtype.sty}{% use microtype if available
  \usepackage[]{microtype}
  \UseMicrotypeSet[protrusion]{basicmath} % disable protrusion for tt fonts
}{}
\makeatletter
\@ifundefined{KOMAClassName}{% if non-KOMA class
  \IfFileExists{parskip.sty}{%
    \usepackage{parskip}
  }{% else
    \setlength{\parindent}{0pt}
    \setlength{\parskip}{6pt plus 2pt minus 1pt}}
}{% if KOMA class
  \KOMAoptions{parskip=half}}
\makeatother
\usepackage{xcolor}
\usepackage[margin=1in]{geometry}
\usepackage{color}
\usepackage{fancyvrb}
\newcommand{\VerbBar}{|}
\newcommand{\VERB}{\Verb[commandchars=\\\{\}]}
\DefineVerbatimEnvironment{Highlighting}{Verbatim}{commandchars=\\\{\}}
% Add ',fontsize=\small' for more characters per line
\usepackage{framed}
\definecolor{shadecolor}{RGB}{248,248,248}
\newenvironment{Shaded}{\begin{snugshade}}{\end{snugshade}}
\newcommand{\AlertTok}[1]{\textcolor[rgb]{0.94,0.16,0.16}{#1}}
\newcommand{\AnnotationTok}[1]{\textcolor[rgb]{0.56,0.35,0.01}{\textbf{\textit{#1}}}}
\newcommand{\AttributeTok}[1]{\textcolor[rgb]{0.13,0.29,0.53}{#1}}
\newcommand{\BaseNTok}[1]{\textcolor[rgb]{0.00,0.00,0.81}{#1}}
\newcommand{\BuiltInTok}[1]{#1}
\newcommand{\CharTok}[1]{\textcolor[rgb]{0.31,0.60,0.02}{#1}}
\newcommand{\CommentTok}[1]{\textcolor[rgb]{0.56,0.35,0.01}{\textit{#1}}}
\newcommand{\CommentVarTok}[1]{\textcolor[rgb]{0.56,0.35,0.01}{\textbf{\textit{#1}}}}
\newcommand{\ConstantTok}[1]{\textcolor[rgb]{0.56,0.35,0.01}{#1}}
\newcommand{\ControlFlowTok}[1]{\textcolor[rgb]{0.13,0.29,0.53}{\textbf{#1}}}
\newcommand{\DataTypeTok}[1]{\textcolor[rgb]{0.13,0.29,0.53}{#1}}
\newcommand{\DecValTok}[1]{\textcolor[rgb]{0.00,0.00,0.81}{#1}}
\newcommand{\DocumentationTok}[1]{\textcolor[rgb]{0.56,0.35,0.01}{\textbf{\textit{#1}}}}
\newcommand{\ErrorTok}[1]{\textcolor[rgb]{0.64,0.00,0.00}{\textbf{#1}}}
\newcommand{\ExtensionTok}[1]{#1}
\newcommand{\FloatTok}[1]{\textcolor[rgb]{0.00,0.00,0.81}{#1}}
\newcommand{\FunctionTok}[1]{\textcolor[rgb]{0.13,0.29,0.53}{\textbf{#1}}}
\newcommand{\ImportTok}[1]{#1}
\newcommand{\InformationTok}[1]{\textcolor[rgb]{0.56,0.35,0.01}{\textbf{\textit{#1}}}}
\newcommand{\KeywordTok}[1]{\textcolor[rgb]{0.13,0.29,0.53}{\textbf{#1}}}
\newcommand{\NormalTok}[1]{#1}
\newcommand{\OperatorTok}[1]{\textcolor[rgb]{0.81,0.36,0.00}{\textbf{#1}}}
\newcommand{\OtherTok}[1]{\textcolor[rgb]{0.56,0.35,0.01}{#1}}
\newcommand{\PreprocessorTok}[1]{\textcolor[rgb]{0.56,0.35,0.01}{\textit{#1}}}
\newcommand{\RegionMarkerTok}[1]{#1}
\newcommand{\SpecialCharTok}[1]{\textcolor[rgb]{0.81,0.36,0.00}{\textbf{#1}}}
\newcommand{\SpecialStringTok}[1]{\textcolor[rgb]{0.31,0.60,0.02}{#1}}
\newcommand{\StringTok}[1]{\textcolor[rgb]{0.31,0.60,0.02}{#1}}
\newcommand{\VariableTok}[1]{\textcolor[rgb]{0.00,0.00,0.00}{#1}}
\newcommand{\VerbatimStringTok}[1]{\textcolor[rgb]{0.31,0.60,0.02}{#1}}
\newcommand{\WarningTok}[1]{\textcolor[rgb]{0.56,0.35,0.01}{\textbf{\textit{#1}}}}
\usepackage{graphicx}
\makeatletter
\def\maxwidth{\ifdim\Gin@nat@width>\linewidth\linewidth\else\Gin@nat@width\fi}
\def\maxheight{\ifdim\Gin@nat@height>\textheight\textheight\else\Gin@nat@height\fi}
\makeatother
% Scale images if necessary, so that they will not overflow the page
% margins by default, and it is still possible to overwrite the defaults
% using explicit options in \includegraphics[width, height, ...]{}
\setkeys{Gin}{width=\maxwidth,height=\maxheight,keepaspectratio}
% Set default figure placement to htbp
\makeatletter
\def\fps@figure{htbp}
\makeatother
\setlength{\emergencystretch}{3em} % prevent overfull lines
\providecommand{\tightlist}{%
  \setlength{\itemsep}{0pt}\setlength{\parskip}{0pt}}
\setcounter{secnumdepth}{-\maxdimen} % remove section numbering
\ifLuaTeX
  \usepackage{selnolig}  % disable illegal ligatures
\fi
\IfFileExists{bookmark.sty}{\usepackage{bookmark}}{\usepackage{hyperref}}
\IfFileExists{xurl.sty}{\usepackage{xurl}}{} % add URL line breaks if available
\urlstyle{same}
\hypersetup{
  pdftitle={Curso pre-congreso Inmunoinformática},
  pdfauthor={Coordinadores: Dr.Otoniel Rodríguez Jorge Dra. Linda Aimara Kempis Calanis},
  hidelinks,
  pdfcreator={LaTeX via pandoc}}

\title{Curso pre-congreso Inmunoinformática}
\author{Coordinadores: Dr.Otoniel Rodríguez Jorge Dra. Linda Aimara
Kempis Calanis}
\date{2025-09-26}

\begin{document}
\maketitle

\hypertarget{librerias-a-utilizar}{%
\subsection{Librerias a utilizar}\label{librerias-a-utilizar}}

\begin{Shaded}
\begin{Highlighting}[]
\FunctionTok{library}\NormalTok{(TxDb.Hsapiens.UCSC.hg38.knownGene)}
\FunctionTok{library}\NormalTok{(ensembldb)}
\FunctionTok{library}\NormalTok{(biomaRt)}
\FunctionTok{library}\NormalTok{(gplots)}
\FunctionTok{library}\NormalTok{(DESeq2)}
\FunctionTok{library}\NormalTok{(pheatmap)}
\FunctionTok{library}\NormalTok{(RColorBrewer)}
\FunctionTok{library}\NormalTok{(tidyverse)}
\FunctionTok{library}\NormalTok{(dplyr)}
\FunctionTok{library}\NormalTok{(clusterProfiler)}
\FunctionTok{library}\NormalTok{(here)}
\end{Highlighting}
\end{Shaded}

\hypertarget{leer-el-archivo-csv}{%
\subsection{Leer el archivo CSV}\label{leer-el-archivo-csv}}

\begin{Shaded}
\begin{Highlighting}[]
\CommentTok{\# Leer el archivo CSV “countsAdultsReal.csv” que está en la carpeta “datos”}
\CommentTok{\# Se usa here::here para construir la ruta al archivo de forma robusta dentro del proyecto}
\NormalTok{counts }\OtherTok{\textless{}{-}} \FunctionTok{read.csv}\NormalTok{(here}\SpecialCharTok{::}\FunctionTok{here}\NormalTok{(}\StringTok{"datos/countsAdultsReal.csv"}\NormalTok{),}\AttributeTok{row.names =} \DecValTok{1}\NormalTok{)}
\CommentTok{\# Mostrar las primeras filas del data frame resultante (útil para verificar que se leyó correctamente)}

\FunctionTok{head}\NormalTok{(counts)}
\end{Highlighting}
\end{Shaded}

\begin{verbatim}
##             Ad1 Ad2 Ad3 Ad1.CD3CD28 Ad2.CD3CD28 Ad3.CD3CD28
## XLOC_000028   4   2   8           1           5           2
## XLOC_000043  40  40  38          22          30          34
## XLOC_000047   7   4   2           2           5           0
## XLOC_000055   0   0   0           0           0           0
## XLOC_000060  50  40  59          28          31          53
## XLOC_000061   5   5   5           4           6           7
\end{verbatim}

\hypertarget{crear-un-metadata}{%
\subsection{Crear un Metadata}\label{crear-un-metadata}}

\begin{Shaded}
\begin{Highlighting}[]
\CommentTok{\# Definir el vector “condition” con las condiciones (factores) correspondientes}
\CommentTok{\# Aquí se usan dos niveles: "Ad SE" y "Ad CD3CD28"}
\NormalTok{condition }\OtherTok{=} \FunctionTok{factor}\NormalTok{(}\FunctionTok{c}\NormalTok{(}\StringTok{"Ad SE"}\NormalTok{,}\StringTok{"Ad SE"}\NormalTok{, }\StringTok{"Ad SE"}\NormalTok{, }\StringTok{"Ad CD3CD28"}\NormalTok{,}\StringTok{"Ad CD3CD28"}\NormalTok{, }\StringTok{"Ad CD3CD28"}\NormalTok{))}

\CommentTok{\# Crear un data.frame “meta” que contendrá esa variable “condition”}
\CommentTok{\# Se asignan los nombres de fila (row.names) usando los nombres de columnas del objeto counts}
\NormalTok{meta }\OtherTok{\textless{}{-}} \FunctionTok{data.frame}\NormalTok{(condition, }\AttributeTok{row.names =} \FunctionTok{colnames}\NormalTok{(counts))}
\FunctionTok{head}\NormalTok{(meta)}
\end{Highlighting}
\end{Shaded}

\begin{verbatim}
##              condition
## Ad1              Ad SE
## Ad2              Ad SE
## Ad3              Ad SE
## Ad1.CD3CD28 Ad CD3CD28
## Ad2.CD3CD28 Ad CD3CD28
## Ad3.CD3CD28 Ad CD3CD28
\end{verbatim}

\begin{Shaded}
\begin{Highlighting}[]
\CommentTok{\#Asegurarse de que los nombres de las filas de “meta”}
\CommentTok{\# coincidan con los nombres de columnas de “counts”}
\FunctionTok{rownames}\NormalTok{(meta) }\SpecialCharTok{==} \FunctionTok{colnames}\NormalTok{(counts)}
\end{Highlighting}
\end{Shaded}

\begin{verbatim}
## [1] TRUE TRUE TRUE TRUE TRUE TRUE
\end{verbatim}

\hypertarget{crear-deseq-dataset}{%
\subsection{Crear DESeq dataset}\label{crear-deseq-dataset}}

\textbf{DESeqDataSetFromMatrix(\ldots)}: función del paquete DESeq2 para
construir un objeto de análisis diferencial de expresión a partir de una
matriz de conteos.

\begin{Shaded}
\begin{Highlighting}[]
\NormalTok{dds }\OtherTok{=} \FunctionTok{DESeqDataSetFromMatrix}\NormalTok{(}\AttributeTok{countData =}\NormalTok{ counts, meta, }\SpecialCharTok{\textasciitilde{}}\NormalTok{ condition)}
\NormalTok{dds }\OtherTok{\textless{}{-}} \FunctionTok{estimateSizeFactors}\NormalTok{(dds)}
\end{Highlighting}
\end{Shaded}

\textbf{estimateSizeFactors(dds)}: función de DESeq2 que calcula los
factores de tamaño (size factors) para cada muestra, para normalizar
diferencias en la profundidad de secuenciación y composición de RNA.

\hypertarget{accede-al-vector-de-factores-de-tamauxf1o-size-factors}{%
\subsection{Accede al vector de factores de tamaño (size
factors)}\label{accede-al-vector-de-factores-de-tamauxf1o-size-factors}}

\begin{Shaded}
\begin{Highlighting}[]
\FunctionTok{sizeFactors}\NormalTok{(dds)}
\end{Highlighting}
\end{Shaded}

\begin{verbatim}
##         Ad1         Ad2         Ad3 Ad1.CD3CD28 Ad2.CD3CD28 Ad3.CD3CD28 
##   1.1561037   0.8907342   0.9627618   0.8908897   1.1851996   0.9805609
\end{verbatim}

\begin{Shaded}
\begin{Highlighting}[]
\FunctionTok{colSums}\NormalTok{(}\FunctionTok{counts}\NormalTok{(dds))}
\end{Highlighting}
\end{Shaded}

\begin{verbatim}
##         Ad1         Ad2         Ad3 Ad1.CD3CD28 Ad2.CD3CD28 Ad3.CD3CD28 
##    26907284    20612097    21921890    20990919    27729938    25010233
\end{verbatim}

\begin{Shaded}
\begin{Highlighting}[]
\FunctionTok{head}\NormalTok{(}\FunctionTok{counts}\NormalTok{(dds))}
\end{Highlighting}
\end{Shaded}

\begin{verbatim}
##             Ad1 Ad2 Ad3 Ad1.CD3CD28 Ad2.CD3CD28 Ad3.CD3CD28
## XLOC_000028   4   2   8           1           5           2
## XLOC_000043  40  40  38          22          30          34
## XLOC_000047   7   4   2           2           5           0
## XLOC_000055   0   0   0           0           0           0
## XLOC_000060  50  40  59          28          31          53
## XLOC_000061   5   5   5           4           6           7
\end{verbatim}

Esta función accede al vector de factores de tamaño (size factors) que
ya ha sido estimado en el objeto dds de tipo DESeqDataSet.

Cada factor de tamaño corresponde a una muestra, y sirve para normalizar
diferencias en profundidad de lectura (library size) u otras variaciones
sistemáticas entre muestras.

\hypertarget{conteos-totales-tras-normalizaciuxf3n-y-transformaciuxf3n-rlog-para-anuxe1lisis-exploratorio}{%
\subsection{Conteos totales tras normalización y transformación rlog
para análisis
exploratorio}\label{conteos-totales-tras-normalizaciuxf3n-y-transformaciuxf3n-rlog-para-anuxe1lisis-exploratorio}}

counts(dds, normalized=TRUE): extrae la matriz de conteos ya
normalizados del objeto dds. Aquí ``normalizados'' significa que los
conteos crudos han sido divididos por los size factors estimados para
las muestras. De esta forma se corrigen (al menos parcialmente)
diferencias de profundidad de lectura / biblioteca entre muestras.

\begin{Shaded}
\begin{Highlighting}[]
\FunctionTok{colSums}\NormalTok{(}\FunctionTok{counts}\NormalTok{(dds, }\AttributeTok{normalized=}\NormalTok{T))}
\end{Highlighting}
\end{Shaded}

\begin{verbatim}
##         Ad1         Ad2         Ad3 Ad1.CD3CD28 Ad2.CD3CD28 Ad3.CD3CD28 
##    23274108    23140570    22769796    23561749    23396851    25506047
\end{verbatim}

rlog(\ldots): es la transformación ``log regularizada'' de DESeq2. Se
usa para transformar los datos (normalizados) al espacio logarítmico,
pero agregando un ``suavizado'' (regularización) para los genes con
conteos bajos, de modo que no se disparen las diferencias relativas para
valores cercanos a cero. Ayuda a estabilizar la varianza cuando los
conteos son bajos.

\begin{Shaded}
\begin{Highlighting}[]
\NormalTok{rld }\OtherTok{\textless{}{-}} \FunctionTok{rlog}\NormalTok{ (dds, }\AttributeTok{blind=}\ConstantTok{TRUE}\NormalTok{)}
\end{Highlighting}
\end{Shaded}

\hypertarget{plot-pca-anuxe1lisis-de-componentes-principales.}{%
\subsection{Plot PCA (análisis de componentes
principales).}\label{plot-pca-anuxe1lisis-de-componentes-principales.}}

\begin{itemize}
\tightlist
\item
  En un gráfico PC1 vs PC2 se ven:
\item
  Cómo se agrupan las muestras (por condición, replicado, lote, etc.).
\item
  Si hay muestras atípicas (outliers).
\item
  Qué tan fuerte es la separación entre condiciones en comparación con
  otras fuentes de variación.
\end{itemize}

\begin{Shaded}
\begin{Highlighting}[]
\FunctionTok{plotPCA}\NormalTok{(rld, }\AttributeTok{intgroup=}\StringTok{"condition"}\NormalTok{)}
\end{Highlighting}
\end{Shaded}

\begin{verbatim}
## using ntop=500 top features by variance
\end{verbatim}

\includegraphics{2.Practica_Deseq2_files/figure-latex/unnamed-chunk-8-1.pdf}

\hypertarget{extraer-la-matriz-de-datos-transformados-con-rlog-desde-el-objeto-rld}{%
\subsection{Extraer la matriz de datos transformados con rlog desde el
objeto
rld}\label{extraer-la-matriz-de-datos-transformados-con-rlog-desde-el-objeto-rld}}

\begin{Shaded}
\begin{Highlighting}[]
\NormalTok{rld\_mat }\OtherTok{\textless{}{-}} \FunctionTok{assay}\NormalTok{(rld)}
\end{Highlighting}
\end{Shaded}

\begin{itemize}
\item
  Una vez extraída la matriz rld\_mat, puedes utilizarla para:
\item
  Análisis exploratorio: Como la realización de análisis de componentes
  principales (PCA) o clustering jerárquico para visualizar y explorar
  las relaciones entre las muestras.
\item
  Visualización: Creación de mapas de calor (heatmaps) para observar la
  expresión génica entre las muestras.
\end{itemize}

\hypertarget{calcular-la-matriz-de-correlaciuxf3n-entre-las-muestras}{%
\subsection{Calcular la matriz de correlación entre las
muestras}\label{calcular-la-matriz-de-correlaciuxf3n-entre-las-muestras}}

Método de correlación: El coeficiente de Pearson mide la relación lineal
entre variables.

\begin{Shaded}
\begin{Highlighting}[]
\NormalTok{rld\_cor }\OtherTok{\textless{}{-}} \FunctionTok{cor}\NormalTok{(rld\_mat)}
\end{Highlighting}
\end{Shaded}

\hypertarget{plot-heatmap}{%
\subsection{Plot heatmap}\label{plot-heatmap}}

\begin{itemize}
\item
  \textbf{pheatmap()} toma como entrada una matriz numérica (por
  ejemplo, una matriz de expresión génica) y genera un mapa de calor
  donde:
\item
  Filas: representan las observaciones (por ejemplo, genes).
\item
  Columnas: representan las variables (por ejemplo, muestras).
\item
  Colores: indican la magnitud de los valores en la matriz, facilitando
  la identificación de patrones y relaciones.
\end{itemize}

\begin{Shaded}
\begin{Highlighting}[]
\FunctionTok{pheatmap}\NormalTok{(rld\_cor)}
\end{Highlighting}
\end{Shaded}

\includegraphics{2.Practica_Deseq2_files/figure-latex/unnamed-chunk-11-1.pdf}

\begin{Shaded}
\begin{Highlighting}[]
\NormalTok{heat.colors }\OtherTok{\textless{}{-}} \FunctionTok{brewer.pal}\NormalTok{(}\DecValTok{6}\NormalTok{, }\StringTok{"Blues"}\NormalTok{)}
\end{Highlighting}
\end{Shaded}

\begin{Shaded}
\begin{Highlighting}[]
\FunctionTok{pheatmap}\NormalTok{(rld\_cor, }\AttributeTok{color =}\NormalTok{ heat.colors, }\AttributeTok{border\_color=}\ConstantTok{NA}\NormalTok{, }\AttributeTok{fontsize =} \DecValTok{10}\NormalTok{, }\AttributeTok{fontsize\_row =} \DecValTok{10}\NormalTok{, }\AttributeTok{height=}\DecValTok{20}\NormalTok{)}
\end{Highlighting}
\end{Shaded}

\includegraphics{2.Practica_Deseq2_files/figure-latex/unnamed-chunk-12-1.pdf}

\hypertarget{generar-gruxe1fico-de-dispersiuxf3n-de-estimaciones}{%
\subsection{Generar gráfico de dispersión de
estimaciones}\label{generar-gruxe1fico-de-dispersiuxf3n-de-estimaciones}}

dds\_results \textless- DESeq(dds): Esta línea ejecuta el análisis de
expresión génica diferencial utilizando el objeto dds, que contiene los
datos de recuento de las muestras. La función DESeq() realiza varios
pasos, incluyendo la normalización de los datos, la estimación de
dispersión y el ajuste del modelo.

\begin{Shaded}
\begin{Highlighting}[]
\NormalTok{dds\_results }\OtherTok{\textless{}{-}} \FunctionTok{DESeq}\NormalTok{(dds)}
\end{Highlighting}
\end{Shaded}

\begin{verbatim}
## using pre-existing size factors
\end{verbatim}

\begin{verbatim}
## estimating dispersions
\end{verbatim}

\begin{verbatim}
## gene-wise dispersion estimates
\end{verbatim}

\begin{verbatim}
## mean-dispersion relationship
\end{verbatim}

\begin{verbatim}
##   Note: levels of factors in the design contain characters other than
##   letters, numbers, '_' and '.'. It is recommended (but not required) to use
##   only letters, numbers, and delimiters '_' or '.', as these are safe characters
##   for column names in R. [This is a message, not a warning or an error]
\end{verbatim}

\begin{verbatim}
## final dispersion estimates
\end{verbatim}

\begin{verbatim}
## fitting model and testing
\end{verbatim}

\begin{Shaded}
\begin{Highlighting}[]
\FunctionTok{plotDispEsts}\NormalTok{(dds\_results)}
\end{Highlighting}
\end{Shaded}

\includegraphics{2.Practica_Deseq2_files/figure-latex/unnamed-chunk-13-1.pdf}
¿Qué muestra el gráfico de dispersión?

El gráfico generado por plotDispEsts() presenta:

Puntos negros: Estimaciones de dispersión por gen.

Línea roja: Ajuste de la relación media-dispersión.

Puntos azules: Estimaciones finales de dispersión utilizadas para las
pruebas estadísticas.

Este gráfico es útil para evaluar si los datos se ajustan bien al modelo
de DESeq2. Se espera que la dispersión disminuya a medida que aumenta la
media de los recuentos. Si observas patrones inusuales o dispersión
elevada en genes con baja expresión, podría indicar problemas en los
datos o la necesidad de ajustar el modelo.

\hypertarget{generar-contraste-y-aplicar-reducciuxf3n-de-lfc}{%
\subsection{Generar contraste y aplicar reducción de
LFC}\label{generar-contraste-y-aplicar-reducciuxf3n-de-lfc}}

\begin{Shaded}
\begin{Highlighting}[]
\DocumentationTok{\#\# Definir el contraste entre condiciones}

\NormalTok{contrast }\OtherTok{\textless{}{-}} \FunctionTok{c}\NormalTok{(}\StringTok{"condition"}\NormalTok{,}\StringTok{"Ad SE"}\NormalTok{,}\StringTok{"Ad CD3CD28"}\NormalTok{)}

\DocumentationTok{\#\# Obtener resultados sin reducción de LFC}

\NormalTok{res1 }\OtherTok{\textless{}{-}} \FunctionTok{results}\NormalTok{(dds\_results,}\AttributeTok{contrast =} \FunctionTok{c}\NormalTok{(}\StringTok{"condition"}\NormalTok{,}\StringTok{"Ad SE"}\NormalTok{,}\StringTok{"Ad CD3CD28"}\NormalTok{), }\AttributeTok{alpha =} \FloatTok{0.05}\NormalTok{)}


\DocumentationTok{\#\# Aplicar reducción de LFC para mejorar la estimación}

\NormalTok{res1 }\OtherTok{\textless{}{-}} \FunctionTok{lfcShrink}\NormalTok{(dds\_results, }\AttributeTok{contrast =} \FunctionTok{c}\NormalTok{(}\StringTok{"condition"}\NormalTok{,}\StringTok{"Ad SE"}\NormalTok{,}\StringTok{"Ad CD3CD28"}\NormalTok{), }\AttributeTok{res=}\NormalTok{res1, }\AttributeTok{type =} \StringTok{"normal"}\NormalTok{)}
\end{Highlighting}
\end{Shaded}

\begin{verbatim}
## using 'normal' for LFC shrinkage, the Normal prior from Love et al (2014).
## 
## Note that type='apeglm' and type='ashr' have shown to have less bias than type='normal'.
## See ?lfcShrink for more details on shrinkage type, and the DESeq2 vignette.
## Reference: https://doi.org/10.1093/bioinformatics/bty895
\end{verbatim}

\begin{Shaded}
\begin{Highlighting}[]
\DocumentationTok{\#\# Resumen de los resultados}
\FunctionTok{summary}\NormalTok{(res1)}
\end{Highlighting}
\end{Shaded}

\begin{verbatim}
## 
## out of 32558 with nonzero total read count
## adjusted p-value < 0.05
## LFC > 0 (up)       : 1992, 6.1%
## LFC < 0 (down)     : 3126, 9.6%
## outliers [1]       : 15, 0.046%
## low counts [2]     : 13155, 40%
## (mean count < 4)
## [1] see 'cooksCutoff' argument of ?results
## [2] see 'independentFiltering' argument of ?results
\end{verbatim}

\hypertarget{generar-gruxe1fico-ma-para-visualizar-genes-diferencialmente-expresados}{%
\subsection{Generar gráfico MA para visualizar genes diferencialmente
expresados}\label{generar-gruxe1fico-ma-para-visualizar-genes-diferencialmente-expresados}}

\begin{Shaded}
\begin{Highlighting}[]
\FunctionTok{plotMA}\NormalTok{(res1, }\AttributeTok{ylim=}\FunctionTok{c}\NormalTok{(}\SpecialCharTok{{-}}\DecValTok{5}\NormalTok{,}\DecValTok{5}\NormalTok{), }\AttributeTok{main=}\StringTok{"Genes diferencialmente expresados"}\NormalTok{)}
\end{Highlighting}
\end{Shaded}

\includegraphics{2.Practica_Deseq2_files/figure-latex/unnamed-chunk-15-1.pdf}

\hypertarget{crear-un-dataframe}{%
\subsection{Crear un dataframe}\label{crear-un-dataframe}}

\begin{Shaded}
\begin{Highlighting}[]
\NormalTok{res }\OtherTok{\textless{}{-}} \FunctionTok{data.frame}\NormalTok{(res1)}
\NormalTok{padj.cutoff }\OtherTok{\textless{}{-}} \FloatTok{0.05}
\NormalTok{lfc.cutoff }\OtherTok{\textless{}{-}} \DecValTok{1}
\end{Highlighting}
\end{Shaded}

\hypertarget{filtrar-y-guardar-genes-diferencialmente-expresados-con-lfc-1}{%
\subsection{Filtrar y guardar genes diferencialmente expresados con LFC
≥
1}\label{filtrar-y-guardar-genes-diferencialmente-expresados-con-lfc-1}}

\begin{Shaded}
\begin{Highlighting}[]
\DocumentationTok{\#\# Convertir los resultados a un tibble y agregar nombres de genes}

\NormalTok{res\_table1 }\OtherTok{\textless{}{-}}\NormalTok{ res }\SpecialCharTok{\%\textgreater{}\%} \FunctionTok{data.frame}\NormalTok{()}\SpecialCharTok{\%\textgreater{}\%}\FunctionTok{rownames\_to\_column}\NormalTok{(}\AttributeTok{var=}\StringTok{"gene"}\NormalTok{) }\SpecialCharTok{\%\textgreater{}\%} \FunctionTok{as\_tibble}\NormalTok{()}

\DocumentationTok{\#\# Filtrar genes con valor ajustado de p (padj) menor o igual al umbral}

\NormalTok{sig1\_padj}\OtherTok{\textless{}{-}}\NormalTok{res }\SpecialCharTok{\%\textgreater{}\%} \FunctionTok{filter}\NormalTok{(padj }\SpecialCharTok{\textless{}=}\NormalTok{padj.cutoff)}

\DocumentationTok{\#\# Filtrar genes con padj \textless{} 0.05 y LFC absoluto \textgreater{} 1}

\NormalTok{sig1 }\OtherTok{\textless{}{-}}\NormalTok{ res\_table1 }\SpecialCharTok{\%\textgreater{}\%} \FunctionTok{filter}\NormalTok{((padj }\SpecialCharTok{\textless{}}\NormalTok{ padj.cutoff) }\SpecialCharTok{\&}\NormalTok{ (}\FunctionTok{abs}\NormalTok{(log2FoldChange) }\SpecialCharTok{\textgreater{}}\NormalTok{ lfc.cutoff)) }\SpecialCharTok{\%\textgreater{}\%} \FunctionTok{arrange}\NormalTok{(}\FunctionTok{desc}\NormalTok{(log2FoldChange))}

\DocumentationTok{\#\# Filtrar genes con padj \textless{} 0.05 y LFC positivo \textgreater{} 1}

\NormalTok{sigA }\OtherTok{\textless{}{-}}\NormalTok{ res\_table1 }\SpecialCharTok{\%\textgreater{}\%} \FunctionTok{filter}\NormalTok{(padj }\SpecialCharTok{\textless{}}\NormalTok{ padj.cutoff }\SpecialCharTok{\&}\NormalTok{ log2FoldChange }\SpecialCharTok{\textgreater{}}\NormalTok{ lfc.cutoff)}

\DocumentationTok{\#\# Filtrar genes con padj \textless{} 0.05 y LFC negativo \textless{} {-}1}

\NormalTok{sigCD3CD28}\OtherTok{\textless{}{-}}\NormalTok{ res\_table1 }\SpecialCharTok{\%\textgreater{}\%} \FunctionTok{filter}\NormalTok{(padj }\SpecialCharTok{\textless{}}\NormalTok{ padj.cutoff }\SpecialCharTok{\&}\NormalTok{ log2FoldChange }\SpecialCharTok{\textless{}} \SpecialCharTok{{-}}\NormalTok{lfc.cutoff)}


\DocumentationTok{\#\# Mostrar dimensiones de los subconjuntos}

\FunctionTok{dim}\NormalTok{(sigA)}
\end{Highlighting}
\end{Shaded}

\begin{verbatim}
## [1] 522   7
\end{verbatim}

\begin{Shaded}
\begin{Highlighting}[]
\FunctionTok{dim}\NormalTok{(sigCD3CD28)}
\end{Highlighting}
\end{Shaded}

\begin{verbatim}
## [1] 1433    7
\end{verbatim}

\begin{Shaded}
\begin{Highlighting}[]
\FunctionTok{dim}\NormalTok{(sig1)}
\end{Highlighting}
\end{Shaded}

\begin{verbatim}
## [1] 1955    7
\end{verbatim}

\begin{Shaded}
\begin{Highlighting}[]
\DocumentationTok{\#\# Mostrar las primeras 100 filas de sigCD3CD28}

\FunctionTok{head}\NormalTok{(sigCD3CD28, }\AttributeTok{n=}\DecValTok{100}\NormalTok{)}
\end{Highlighting}
\end{Shaded}

\begin{verbatim}
## # A tibble: 100 x 7
##    gene        baseMean log2FoldChange lfcSE  stat       pvalue        padj
##    <chr>          <dbl>          <dbl> <dbl> <dbl>        <dbl>       <dbl>
##  1 XLOC_000986     8.09          -1.55 0.535 -2.83 0.00469      0.0212     
##  2 XLOC_000988     5.97          -1.93 0.553 -3.33 0.000882     0.00504    
##  3 XLOC_002947     5.39          -1.57 0.552 -2.73 0.00637      0.0274     
##  4 XLOC_003913    34.7           -2.37 0.423 -5.49 0.0000000408 0.000000613
##  5 XLOC_003921     7.41          -2.39 0.553 -4.33 0.0000151    0.000137   
##  6 XLOC_004229    13.9           -1.36 0.495 -2.71 0.00664      0.0283     
##  7 XLOC_006357     3.72          -1.40 0.545 -2.64 0.00826      0.0342     
##  8 XLOC_006358    25.0           -1.68 0.423 -3.92 0.0000884    0.000666   
##  9 XLOC_006572     4.82          -1.69 0.550 -3.01 0.00262      0.0128     
## 10 XLOC_007590    48.5           -1.58 0.414 -3.79 0.000148     0.00106    
## # i 90 more rows
\end{verbatim}

\begin{Shaded}
\begin{Highlighting}[]
\DocumentationTok{\#\# Guardar resultados en archivos de texto}

\FunctionTok{write.table}\NormalTok{(sigA,}\AttributeTok{file =} \FunctionTok{here}\NormalTok{(}\StringTok{"resultados\_tablas/Ad\_SE.txt"}\NormalTok{), }\AttributeTok{sep =} \StringTok{"}\SpecialCharTok{\textbackslash{}t}\StringTok{"}\NormalTok{, }\AttributeTok{col.names =} \ConstantTok{TRUE}\NormalTok{)}

\FunctionTok{write.table}\NormalTok{(sigCD3CD28, }\AttributeTok{file=} \FunctionTok{here}\NormalTok{(}\StringTok{"resultados\_tablas/Ad\_CD3CD28.txt"}\NormalTok{), }\AttributeTok{sep =} \StringTok{"}\SpecialCharTok{\textbackslash{}t}\StringTok{"}\NormalTok{, }\AttributeTok{col.names =} \ConstantTok{TRUE}\NormalTok{)}
\end{Highlighting}
\end{Shaded}

\hypertarget{anuxe1lisis-de-enriquecimiento-de-conjuntos-de-genes-con-clusterprofiler}{%
\subsection{Análisis de enriquecimiento de conjuntos de genes con
ClusterProfiler}\label{anuxe1lisis-de-enriquecimiento-de-conjuntos-de-genes-con-clusterprofiler}}

\href{https://github.com/YuLab-SMU/clusterProfiler}{\textbf{ClusterProfiler}}

\hypertarget{librerias-parte-2}{%
\subsection{Librerias parte 2}\label{librerias-parte-2}}

Si ya tienes una lista de genes diferencialmente expresados (DEGs) en
formato de símbolos de genes, puedes utilizar la función bitr() del
paquete clusterProfiler para convertir estos identificadores a ENTREZID.
Asegúrate de tener cargado el paquete org.Hs.eg.db para acceder a la
base de datos de anotación de genes humanos

\begin{Shaded}
\begin{Highlighting}[]
\FunctionTok{library}\NormalTok{(clusterProfiler)}
\FunctionTok{library}\NormalTok{(org.Hs.eg.db)}
\FunctionTok{library}\NormalTok{(ggplot2)}
\CommentTok{\#packageDescription("clusterProfiler")}
\end{Highlighting}
\end{Shaded}

\begin{itemize}
\tightlist
\item
  Necesitamos los entrezgenid de las variable
\end{itemize}

\begin{Shaded}
\begin{Highlighting}[]
\CommentTok{\# Convertir símbolos de genes a ENTREZID}
\NormalTok{GDE\_SE\_ENTREZ }\OtherTok{\textless{}{-}} \FunctionTok{bitr}\NormalTok{(sigA}\SpecialCharTok{$}\NormalTok{gene, }\AttributeTok{fromType =} \StringTok{"SYMBOL"}\NormalTok{, }\AttributeTok{toType =} \StringTok{"ENTREZID"}\NormalTok{, }\AttributeTok{OrgDb =}\NormalTok{ org.Hs.eg.db)}
\end{Highlighting}
\end{Shaded}

\begin{verbatim}
## 'select()' returned 1:1 mapping between keys and columns
\end{verbatim}

\begin{verbatim}
## Warning in bitr(sigA$gene, fromType = "SYMBOL", toType = "ENTREZID", OrgDb =
## org.Hs.eg.db): 31.23% of input gene IDs are fail to map...
\end{verbatim}

\begin{Shaded}
\begin{Highlighting}[]
\NormalTok{GDE\_ES\_ENTREZ }\OtherTok{\textless{}{-}} \FunctionTok{bitr}\NormalTok{(sigCD3CD28}\SpecialCharTok{$}\NormalTok{gene, }\AttributeTok{fromType =} \StringTok{"SYMBOL"}\NormalTok{, }\AttributeTok{toType =} \StringTok{"ENTREZID"}\NormalTok{, }\AttributeTok{OrgDb =}\NormalTok{ org.Hs.eg.db)}
\end{Highlighting}
\end{Shaded}

\begin{verbatim}
## 'select()' returned 1:1 mapping between keys and columns
\end{verbatim}

\begin{verbatim}
## Warning in bitr(sigCD3CD28$gene, fromType = "SYMBOL", toType = "ENTREZID", :
## 11.17% of input gene IDs are fail to map...
\end{verbatim}

\begin{Shaded}
\begin{Highlighting}[]
\FunctionTok{head}\NormalTok{(GDE\_SE\_ENTREZ)}
\end{Highlighting}
\end{Shaded}

\begin{verbatim}
##     SYMBOL ENTREZID
## 72  TAS1R3    83756
## 73   MXRA8    54587
## 74   MEGF6     1953
## 75   AJAP1    55966
## 78 LDLRAP1    26119
## 79  CRYBG2    55057
\end{verbatim}

\begin{Shaded}
\begin{Highlighting}[]
\FunctionTok{head}\NormalTok{(GDE\_ES\_ENTREZ)}
\end{Highlighting}
\end{Shaded}

\begin{verbatim}
##      SYMBOL ENTREZID
## 45 TNFRSF18     8784
## 46  TNFRSF4     7293
## 47   ATAD3A    55210
## 48 TNFRSF14     8764
## 49     TP73     7161
## 50   RNF207   388591
\end{verbatim}

\begin{Shaded}
\begin{Highlighting}[]
\NormalTok{SE }\OtherTok{\textless{}{-}}\NormalTok{sigA}\SpecialCharTok{$}\NormalTok{gene}
\NormalTok{ES }\OtherTok{\textless{}{-}}\NormalTok{sigCD3CD28}\SpecialCharTok{$}\NormalTok{gene}
\NormalTok{GDE\_SE }\OtherTok{=} \FunctionTok{bitr}\NormalTok{(SE, }\AttributeTok{fromType=}\StringTok{"SYMBOL"}\NormalTok{, }\AttributeTok{toType=}\StringTok{"ENTREZID"}\NormalTok{, }\AttributeTok{OrgDb=}\StringTok{"org.Hs.eg.db"}\NormalTok{)}
\end{Highlighting}
\end{Shaded}

\begin{verbatim}
## 'select()' returned 1:1 mapping between keys and columns
\end{verbatim}

\begin{verbatim}
## Warning in bitr(SE, fromType = "SYMBOL", toType = "ENTREZID", OrgDb =
## "org.Hs.eg.db"): 31.23% of input gene IDs are fail to map...
\end{verbatim}

\begin{Shaded}
\begin{Highlighting}[]
\FunctionTok{head}\NormalTok{(GDE\_SE)}
\end{Highlighting}
\end{Shaded}

\begin{verbatim}
##     SYMBOL ENTREZID
## 72  TAS1R3    83756
## 73   MXRA8    54587
## 74   MEGF6     1953
## 75   AJAP1    55966
## 78 LDLRAP1    26119
## 79  CRYBG2    55057
\end{verbatim}

\begin{Shaded}
\begin{Highlighting}[]
\NormalTok{GDE\_ES }\OtherTok{=} \FunctionTok{bitr}\NormalTok{(ES, }\AttributeTok{fromType=}\StringTok{"SYMBOL"}\NormalTok{, }\AttributeTok{toType=}\StringTok{"ENTREZID"}\NormalTok{, }\AttributeTok{OrgDb=}\StringTok{"org.Hs.eg.db"}\NormalTok{)}
\end{Highlighting}
\end{Shaded}

\begin{verbatim}
## 'select()' returned 1:1 mapping between keys and columns
\end{verbatim}

\begin{verbatim}
## Warning in bitr(ES, fromType = "SYMBOL", toType = "ENTREZID", OrgDb =
## "org.Hs.eg.db"): 11.17% of input gene IDs are fail to map...
\end{verbatim}

\begin{Shaded}
\begin{Highlighting}[]
\FunctionTok{head}\NormalTok{(GDE\_ES)}
\end{Highlighting}
\end{Shaded}

\begin{verbatim}
##      SYMBOL ENTREZID
## 45 TNFRSF18     8784
## 46  TNFRSF4     7293
## 47   ATAD3A    55210
## 48 TNFRSF14     8764
## 49     TP73     7161
## 50   RNF207   388591
\end{verbatim}

\begin{Shaded}
\begin{Highlighting}[]
\NormalTok{GDE\_SE\_ENTREZ }\OtherTok{\textless{}{-}}\NormalTok{GDE\_SE}\SpecialCharTok{$}\NormalTok{ENTREZID}
\NormalTok{GDE\_ES\_ENTREZ }\OtherTok{\textless{}{-}}\NormalTok{GDE\_ES}\SpecialCharTok{$}\NormalTok{ENTREZID}
\end{Highlighting}
\end{Shaded}

\hypertarget{anuxe1lisis-de-enriquecimiento-funcional-comparativo-con-comparecluster}{%
\section{Análisis de enriquecimiento funcional comparativo con
compareCluster}\label{anuxe1lisis-de-enriquecimiento-funcional-comparativo-con-comparecluster}}

\begin{itemize}
\tightlist
\item
  El paquete clusterProfiler en R ofrece la función compareCluster()
  para comparar perfiles funcionales de diferentes conjuntos de genes.
  Esta herramienta es útil para identificar diferencias y similitudes en
  las funciones biológicas entre distintos grupos de genes, como los
  genes diferencialmente expresados en diferentes condiciones
  experimentales.
\end{itemize}

\begin{Shaded}
\begin{Highlighting}[]
\DocumentationTok{\#\# Comparar perfiles funcionales de conjuntos de genes mediante análisis de enriquecimiento GO}
\NormalTok{mi\_lista }\OtherTok{\textless{}{-}} \FunctionTok{list}\NormalTok{(}\StringTok{"SE\_ADULTOS"} \OtherTok{=}\NormalTok{ GDE\_SE\_ENTREZ, }\StringTok{"ES\_Adultos"} \OtherTok{=}\NormalTok{ GDE\_ES\_ENTREZ)}

\DocumentationTok{\#\# Realizar el análisis de enriquecimiento GO para cada conjunto de genes}
\NormalTok{ck1 }\OtherTok{\textless{}{-}} \FunctionTok{compareCluster}\NormalTok{(}\AttributeTok{geneCluster =}\NormalTok{ mi\_lista, }\AttributeTok{fun =} \StringTok{"enrichGO"}\NormalTok{, }\AttributeTok{OrgDb =} \StringTok{"org.Hs.eg.db"}\NormalTok{, }\AttributeTok{ont =} \StringTok{"BP"}\NormalTok{)}

\DocumentationTok{\#\# Guardar los resultados en un archivo PDF}

\FunctionTok{head}\NormalTok{(}\FunctionTok{as.data.frame}\NormalTok{(ck1))  }\CommentTok{\# Mostrar las primeras filas de los resultados}
\end{Highlighting}
\end{Shaded}

\begin{verbatim}
##      Cluster         ID                               Description GeneRatio
## 1 SE_ADULTOS GO:0061448             connective tissue development    18/293
## 2 ES_Adultos GO:0006260                           DNA replication   79/1136
## 3 ES_Adultos GO:0006261             DNA-templated DNA replication   60/1136
## 4 ES_Adultos GO:0007059                    chromosome segregation   97/1136
## 5 ES_Adultos GO:0000280                          nuclear division   93/1136
## 6 ES_Adultos GO:1901987 regulation of cell cycle phase transition   94/1136
##     BgRatio       pvalue     p.adjust       qvalue
## 1 285/18870 5.491646e-07 1.877594e-03 1.706457e-03
## 2 278/18870 1.895580e-32 9.868245e-29 7.689187e-29
## 3 161/18870 3.709867e-32 9.868245e-29 7.689187e-29
## 4 424/18870 3.360736e-31 5.959706e-28 4.643712e-28
## 5 441/18870 4.315619e-27 5.739773e-24 4.472344e-24
## 6 456/18870 1.368659e-26 1.456254e-23 1.134691e-23
##                                                                                                                                                                                                                                                                                                                                                                                                                                                                                                                                                                     geneID
## 1                                                                                                                                                                                                                                                                                                                                                                                                                                                                            4774/3570/57326/8609/7476/5593/7045/1302/5467/55790/6653/7480/30812/7067/9241/8313/5155/89766
## 2                                                                                                                             378708/4678/4998/51514/6241/348654/4175/580/10293/10721/4171/5984/604/890/51053/3159/4172/5558/253714/6119/5982/4176/157570/84296/4173/9134/79075/4796/9401/80010/6240/79733/374393/2237/23649/10714/1111/55388/983/1763/5763/57379/1017/8914/144455/5985/8099/675/5983/5427/11169/5888/1854/9768/80119/54962/90381/641/116028/23594/64785/51659/81620/79915/990/672/146956/7083/10189/201254/5932/10036/4784/898/10856/8318/4174/29980/8208
## 3                                                                                                                                                                                                                                     378708/4998/348654/4175/580/10293/10721/4171/5984/604/51053/3159/4172/5558/253714/5982/4176/84296/4173/9134/79075/4796/9401/6240/79733/2237/23649/10714/55388/1763/57379/1017/8914/144455/5985/8099/675/5983/5427/11169/5888/54962/90381/641/23594/64785/51659/81620/79915/990/672/146956/7083/10189/201254/5932/898/8318/4174/29980
## 4 378708/55143/991/11004/6491/84722/83540/259266/9928/4751/1063/348654/79172/23397/699/57405/51776/57520/151246/55355/151648/4292/5359/10051/1894/64151/1062/4085/84057/79682/9319/891/64946/9232/203068/3833/55166/7272/387103/54892/157313/157570/2669/9134/401541/10592/10783/81930/113130/11130/983/3832/4288/29127/9700/79023/9735/5901/221150/440145/9787/8110/7517/701/90417/57082/11339/51203/9133/55055/9493/9055/116028/5347/55839/81620/83903/9212/10615/990/7153/29893/7283/672/146909/3837/348235/83990/332/201254/10403/220134/147841/256126/898/11144/79019
## 5                                        378708/2827/55143/991/11004/8438/84722/83540/259266/9928/4751/1063/348654/23397/699/57405/151648/4292/10051/64151/1062/4085/84057/9319/891/995/9232/7124/3833/7272/54443/54892/157313/9134/286151/80010/1164/10592/81930/113130/1111/84930/11130/983/3832/4288/10635/29127/9700/3479/9735/5901/675/440145/10419/9787/3306/7517/701/90417/57082/5888/51203/9133/55055/9493/9055/9088/5347/55512/81620/2175/83903/9212/10615/7153/29893/7283/146909/3837/146956/83990/8877/332/201254/10403/147841/256126/898/27338/5902/3976/11144
## 6                                                            26270/63967/55143/991/6491/1031/4998/4582/83540/9928/8444/25896/51514/1063/3398/6241/348654/699/57405/51776/57520/580/993/83666/151636/1062/4085/9319/7015/891/995/8870/1026/7272/54443/1021/2146/1029/10783/6240/7465/113130/595/5962/1111/11130/983/1763/894/9700/1017/79035/8914/1019/144455/9735/675/7027/5720/5721/9787/4331/3306/8110/7517/701/57082/5888/79968/54962/55055/1543/90381/641/9088/5347/81620/83903/9212/79915/990/672/146956/83990/332/10403/5932/596/147841/25959/22809/8318/51512/29980
##   Count
## 1    18
## 2    79
## 3    60
## 4    97
## 5    93
## 6    94
\end{verbatim}

\begin{Shaded}
\begin{Highlighting}[]
\NormalTok{x }\OtherTok{\textless{}{-}} \FunctionTok{as.data.frame}\NormalTok{(ck1)}
\FunctionTok{write.table}\NormalTok{(x, }\AttributeTok{file =} \FunctionTok{here}\NormalTok{(}\StringTok{"resultados\_tablas/listas\_enrichGO\_PE.txt"}\NormalTok{), }\AttributeTok{sep =} \StringTok{"}\SpecialCharTok{\textbackslash{}t}\StringTok{"}\NormalTok{, }\AttributeTok{col.names =} \ConstantTok{TRUE}\NormalTok{)}
\FunctionTok{head}\NormalTok{(ck1)  }\CommentTok{\# Ver los resultados en una ventana interactiva}
\end{Highlighting}
\end{Shaded}

\begin{verbatim}
##      Cluster         ID                               Description GeneRatio
## 1 SE_ADULTOS GO:0061448             connective tissue development    18/293
## 2 ES_Adultos GO:0006260                           DNA replication   79/1136
## 3 ES_Adultos GO:0006261             DNA-templated DNA replication   60/1136
## 4 ES_Adultos GO:0007059                    chromosome segregation   97/1136
## 5 ES_Adultos GO:0000280                          nuclear division   93/1136
## 6 ES_Adultos GO:1901987 regulation of cell cycle phase transition   94/1136
##     BgRatio       pvalue     p.adjust       qvalue
## 1 285/18870 5.491646e-07 1.877594e-03 1.706457e-03
## 2 278/18870 1.895580e-32 9.868245e-29 7.689187e-29
## 3 161/18870 3.709867e-32 9.868245e-29 7.689187e-29
## 4 424/18870 3.360736e-31 5.959706e-28 4.643712e-28
## 5 441/18870 4.315619e-27 5.739773e-24 4.472344e-24
## 6 456/18870 1.368659e-26 1.456254e-23 1.134691e-23
##                                                                                                                                                                                                                                                                                                                                                                                                                                                                                                                                                                     geneID
## 1                                                                                                                                                                                                                                                                                                                                                                                                                                                                            4774/3570/57326/8609/7476/5593/7045/1302/5467/55790/6653/7480/30812/7067/9241/8313/5155/89766
## 2                                                                                                                             378708/4678/4998/51514/6241/348654/4175/580/10293/10721/4171/5984/604/890/51053/3159/4172/5558/253714/6119/5982/4176/157570/84296/4173/9134/79075/4796/9401/80010/6240/79733/374393/2237/23649/10714/1111/55388/983/1763/5763/57379/1017/8914/144455/5985/8099/675/5983/5427/11169/5888/1854/9768/80119/54962/90381/641/116028/23594/64785/51659/81620/79915/990/672/146956/7083/10189/201254/5932/10036/4784/898/10856/8318/4174/29980/8208
## 3                                                                                                                                                                                                                                     378708/4998/348654/4175/580/10293/10721/4171/5984/604/51053/3159/4172/5558/253714/5982/4176/84296/4173/9134/79075/4796/9401/6240/79733/2237/23649/10714/55388/1763/57379/1017/8914/144455/5985/8099/675/5983/5427/11169/5888/54962/90381/641/23594/64785/51659/81620/79915/990/672/146956/7083/10189/201254/5932/898/8318/4174/29980
## 4 378708/55143/991/11004/6491/84722/83540/259266/9928/4751/1063/348654/79172/23397/699/57405/51776/57520/151246/55355/151648/4292/5359/10051/1894/64151/1062/4085/84057/79682/9319/891/64946/9232/203068/3833/55166/7272/387103/54892/157313/157570/2669/9134/401541/10592/10783/81930/113130/11130/983/3832/4288/29127/9700/79023/9735/5901/221150/440145/9787/8110/7517/701/90417/57082/11339/51203/9133/55055/9493/9055/116028/5347/55839/81620/83903/9212/10615/990/7153/29893/7283/672/146909/3837/348235/83990/332/201254/10403/220134/147841/256126/898/11144/79019
## 5                                        378708/2827/55143/991/11004/8438/84722/83540/259266/9928/4751/1063/348654/23397/699/57405/151648/4292/10051/64151/1062/4085/84057/9319/891/995/9232/7124/3833/7272/54443/54892/157313/9134/286151/80010/1164/10592/81930/113130/1111/84930/11130/983/3832/4288/10635/29127/9700/3479/9735/5901/675/440145/10419/9787/3306/7517/701/90417/57082/5888/51203/9133/55055/9493/9055/9088/5347/55512/81620/2175/83903/9212/10615/7153/29893/7283/146909/3837/146956/83990/8877/332/201254/10403/147841/256126/898/27338/5902/3976/11144
## 6                                                            26270/63967/55143/991/6491/1031/4998/4582/83540/9928/8444/25896/51514/1063/3398/6241/348654/699/57405/51776/57520/580/993/83666/151636/1062/4085/9319/7015/891/995/8870/1026/7272/54443/1021/2146/1029/10783/6240/7465/113130/595/5962/1111/11130/983/1763/894/9700/1017/79035/8914/1019/144455/9735/675/7027/5720/5721/9787/4331/3306/8110/7517/701/57082/5888/79968/54962/55055/1543/90381/641/9088/5347/81620/83903/9212/79915/990/672/146956/83990/332/10403/5932/596/147841/25959/22809/8318/51512/29980
##   Count
## 1    18
## 2    79
## 3    60
## 4    97
## 5    93
## 6    94
\end{verbatim}

\begin{Shaded}
\begin{Highlighting}[]
\DocumentationTok{\#\# Visualizar los resultados mediante un gráfico de puntos}

\FunctionTok{dotplot}\NormalTok{(ck1, }\AttributeTok{showCategory =} \DecValTok{10}\NormalTok{, }\AttributeTok{font.size =} \FloatTok{10.5}\NormalTok{) }\SpecialCharTok{+} \FunctionTok{ggtitle}\NormalTok{(}\StringTok{"Análisis de enriquecimiento funcional"}\NormalTok{)}
\end{Highlighting}
\end{Shaded}

\includegraphics{2.Practica_Deseq2_files/figure-latex/unnamed-chunk-21-1.pdf}

\hypertarget{volcano-plot}{%
\subsection{Volcano plot}\label{volcano-plot}}

\href{https://bioconductor.org/packages/devel/bioc/vignettes/EnhancedVolcano/inst/doc/EnhancedVolcano.html}{\textbf{EnhancedVolcano}}

\begin{itemize}
\item
  Un volcano plot es una herramienta visual utilizada en análisis de
  expresión diferencial para representar la relación entre la magnitud
  del cambio (fold change) y la significancia estadística (valor p) de
  cada gen. En este gráfico:
\item
  El eje x muestra el log₂ del fold change (log₂FC), indicando la
  magnitud del cambio en la expresión génica entre dos condiciones.
\item
  El eje y muestra el valor p ajustado o el valor p negativo logaritmado
  (-log₁₀(p)), representando la significancia estadística del cambio
  observado.
\item
  Los puntos en las esquinas superiores del gráfico suelen representar
  genes que son tanto estadísticamente significativos como
  biológicamente relevantes, es decir, aquellos con un gran cambio en la
  expresión y un valor p bajo.
\end{itemize}

\begin{Shaded}
\begin{Highlighting}[]
\FunctionTok{library}\NormalTok{(EnhancedVolcano)}
\end{Highlighting}
\end{Shaded}

\begin{verbatim}
## Loading required package: ggrepel
\end{verbatim}

\begin{Shaded}
\begin{Highlighting}[]
\CommentTok{\# Generar el volcano plot}

\FunctionTok{EnhancedVolcano}\NormalTok{(res1,}
    \AttributeTok{lab =} \FunctionTok{rownames}\NormalTok{(res1),}
    \AttributeTok{x =} \StringTok{\textquotesingle{}log2FoldChange\textquotesingle{}}\NormalTok{,}
    \AttributeTok{y =} \StringTok{\textquotesingle{}pvalue\textquotesingle{}}\NormalTok{)}
\end{Highlighting}
\end{Shaded}

\includegraphics{2.Practica_Deseq2_files/figure-latex/unnamed-chunk-22-1.pdf}
\#\# Referencias

\href{https://bioconductor.org/packages/devel/bioc/vignettes/DESeq2/inst/doc/DESeq2.html}{Deseq2}

\end{document}
